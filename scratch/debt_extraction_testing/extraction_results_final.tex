\documentclass[11pt,letterpaper]{article}
\usepackage[margin=0.8in]{geometry}
\usepackage{listings}
\usepackage{xcolor}
\usepackage{fancyhdr}
\usepackage{hyperref}
\usepackage{tcolorbox}
\usepackage{fontspec}
\setmonofont{Courier New}

% JSON syntax highlighting
\definecolor{json-delim}{RGB}{20,105,176}
\definecolor{json-string}{RGB}{163,21,21}
\definecolor{json-number}{RGB}{0,128,0}
\definecolor{json-property}{RGB}{0,0,139}
\definecolor{background}{RGB}{248,248,248}

\lstdefinelanguage{json}{
    basicstyle=\footnotesize\ttfamily,
    numbers=left,
    numberstyle=\tiny\color{gray},
    stepnumber=1,
    numbersep=8pt,
    showstringspaces=false,
    breaklines=true,
    frame=lines,
    backgroundcolor=\color{background},
    string=[s]{"}{"},
    stringstyle=\color{json-string},
    literate=
        *{0}{{{\color{json-number}0}}}{1}
        {1}{{{\color{json-number}1}}}{1}
        {2}{{{\color{json-number}2}}}{1}
        {3}{{{\color{json-number}3}}}{1}
        {4}{{{\color{json-number}4}}}{1}
        {5}{{{\color{json-number}5}}}{1}
        {6}{{{\color{json-number}6}}}{1}
        {7}{{{\color{json-number}7}}}{1}
        {8}{{{\color{json-number}8}}}{1}
        {9}{{{\color{json-number}9}}}{1}
        {:}{{{\color{json-delim}:}}}{1}
        {,}{{{\color{json-delim},}}}{1}
        {\{}{{{\color{json-delim}\{}}}{1}
        {\}}{{{\color{json-delim}\}}}}{1}
        {[}{{{\color{json-delim}[}}}{1}
        {]}{{{\color{json-delim}]}}}{1}
        {true}{{{\color{json-property}true}}}{4}
        {false}{{{\color{json-property}false}}}{5}
        {null}{{{\color{gray}null}}}{4},
}

\lstdefinestyle{prompt}{
    basicstyle=\tiny\ttfamily,
    frame=single,
    backgroundcolor=\color{gray!5},
    breaklines=true,
    breakatwhitespace=true,
    numbers=none
}

\title{\textbf{Debt Document Extraction Using GPT-4.1-nano}}
\author{Research Notes}
\date{August 2025}

\pagestyle{fancy}
\fancyhf{}
\rhead{\thepage}
\cfoot{}

\begin{document}

\maketitle

\section{Summary}

Tested GPT-4.1-nano on debt document extraction. Used natural language prompt instead of JSON schema. Model handles both simple loans and complex syndicated facilities.

Key findings:
\begin{itemize}
\item Numbers come out clean (no text, no commas)
\item Interest rates convert to basis points correctly
\item Dates standardize to YYYY-MM-DD
\item Model doesn't make stuff up - uses null when info not there
\end{itemize}

\section{The Prompt}

Below is the full extraction prompt. It's long (343 lines) but covers everything - parties, pricing, covenants, etc. Main thing is telling the model exactly how to format numbers.

\input{full_prompt_section}

\newpage
\section{Test 1: Simple Commercial Loan}

Ran extraction on RF Monolithics loan docs. Basic \$900k term loan from Viewpoint Bank. 

Results:
\begin{itemize}
\item Got the principal right: 900000 (not "900,000" or "\$900k")
\item Interest rate: Prime + 1\% extracted as spread\_bps: 100
\item Only 2 parties, correctly identified as borrower and lender
\item No hallucinated covenants or fees
\end{itemize}

Full JSON output:

\lstinputlisting[language=json]{extractions/rf_monolithics_clean/extracted_terms.json}

\newpage
\section{Test 2: Syndicated Credit Agreement}

This one's more complex - \$171M facility with multiple lenders, different loan types (revolver, term, L/C).

What worked:
\begin{itemize}
\item Picked up all 7 parties with correct roles
\item Found all facility types and amounts
\item Got the guarantor (WRC Media Inc.)
\item Events of default extracted properly
\end{itemize}

Note: Some amounts show as null because document was truncated for testing. Full doc would have complete numbers.

Full JSON output:

\lstinputlisting[language=json]{extractions/syndicated_clean/extracted_terms.json}

\section{Conclusion}

The extraction works. Same prompt handles everything from simple notes to complex syndicated deals. Output is consistent JSON that can go straight into a database.

Main advantage: no need to write different schemas for each document type. The natural language instructions adapt to whatever's in the document.

\end{document}