\documentclass[11pt]{article}
\usepackage[margin=1in]{geometry}
\usepackage{booktabs}
\usepackage{longtable}
\usepackage{array}
\usepackage{xcolor}
\usepackage{colortbl}
\usepackage{hyperref}

\title{Credit Agreement Extractions with Definitions}
\date{\today}

\begin{document}
\maketitle

\tableofcontents
\newpage

\section*{Overview}
This document presents structured extractions from credit agreements using an enhanced prompt that captures definitions alongside economic terms. Definitions are \colorbox{yellow}{highlighted in yellow} throughout the document.

\newpage

\section{Simple Credit Agreement}

\subsection{Document Overview}
\begin{itemize}
\item \textbf{Document Type:} credit agreement
\item \textbf{Effective Date:} 2024-01-01
\end{itemize}

\subsection{Parties and Commitments}
\subsubsection{Parties}
\begin{tabular}{|l|l|l|}
\hline
\textbf{Role} & \textbf{Name} & \textbf{Jurisdiction} \\
\hline
borrower & Acme Corporation & N/A \\
lender & Widget Bank, N.A. & N/A \\
\hline
\end{tabular}

\subsubsection{Credit Facilities}
\begin{itemize}
\item \textbf{REVOLVING CREDIT FACILITY:} \$12,500,000 USD
\end{itemize}

\subsection{Pricing Terms}
\subsubsection{Base Interest Rate}
\begin{itemize}
\item \textbf{Rate Type:} floating
\item \textbf{Benchmark:} SOFR
\item \textbf{Spread:} 200 basis points
\end{itemize}

\subsection{Administrative Terms}
\begin{itemize}
\item \textbf{Governing Law:} New York
\end{itemize}

\subsection{Summary of Definitions Captured}
This extraction captured \textbf{0 definitions} to explain the economic terms.

\newpage

\end{document}